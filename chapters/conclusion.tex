\chapter{Conclusion}

In my thesis research, I have evaluated the performance of current state-of-the-art computational protein structure prediction and design methods that estimate the energetic effects of mutations, design protein sequences, and predict the structure of protein loops. The curated benchmark data that I assembled to evaluate each of these methods will also be of use for future benchmarking of Rosetta and non-Rosetta protocols.

I have advanced the ability of the Rosetta computational protein structure prediction and design software to more accurately represent proteins with conformational flexibility using ensemble-based approaches.
This ensemble-based sampling approach has enabled improved performance in calculations of change in binding free energy post-mutation, particularly for cases of small-to-large mutations that were difficult to model with previous methods.

I look forward to the continued application of these new methods to the advancement of our knowledge of biology. I hope that my contribution to science will now join the greater body of work generated by countless others around the world, enabling science to continue to advance the health and well-being of humanity.